\documentclass[a4paper]{article}

\usepackage[utf8]{inputenc}
\usepackage[T1]{fontenc}
\usepackage{textcomp}
\usepackage{times}
\usepackage{amsmath, amssymb}

\pdfsuppresswarningpagegroup=1

\begin{document}

	\title{Chemistry Grade 10Th}
	\author{Anonymous}
	\date{\today}
	\maketitle

	\section{Basic Theory}
	\subsection{Atom}
	The smallest particle of matter, no independent existence.
	\subsubsection{Constituents of Atom}
	\begin{table}[htpb]
		\centering
		\caption{Constituents of Atom}
		\label{tab:label}
		\begin{tabular}{ |c|c|c| }
			\hline
			Constituent & Mass & Charge \\
			\hline
			Protons & $1.672 * 10^-27 kg$ & $+1e$ \\
			\hline
			Neutrons & $1.674 * 10^-27 kg$ & $+1e$ \\
			\hline
			Electrons & $9.10 * 10^-31 kg$ & $0e$ \\
			\hline
		\end{tabular}
	\end{table}

	\subsubsection{Properties}
	\begin{itemize}
		\item Atomic Radius: distance between center of nucleus of atom and its outermost shell

			$\vec{{Period}}$ - decreases; \it{Inert gases have bigger atomic raidus than halogens}

			Group $\downarrow$ - increases

		\item Nuclear Charge: positive charge present in the nucleus of atom, equal to number of protons, i.e., the atomic number

			always increases

		\item Valency: combining capacity of atom or the number of electrons lost, gained or shared by the atom

			$\vec{{Period}}$ - increases till 4, then decreases

			Group $\downarrow$ - remains same

		\item Metallic Character: tendency of losing valence electrons

			$\vec{{Period}}$ - decreases

			Group $\downarrow$ - increases

		\item Non-Metallic Character: tendency of gaining valence electrons

			$\vec{{Period}}$ - increases

			Group $\downarrow$ - decreases

		\item Chemical Reactivity: tendency of losing or gaining electrons

			$\vec{{Period}}$ - decreases then increases

			Group $\downarrow$ - increases

		\item Ionization Potential: The energy required to \underline{remove} an electron from a \underline{neutral isolated gaseous} atom and convert it into a positively charged gaseous ion

			Unit: $kJ mol^-1$

			$\vec{{Period}}$ - increases; \it{He is the highest, Cs is the lowest, Fr is not determined correctly as it is radioactive}

			Group $\downarrow$ - decreases

		\item Electron Affinity: The amount of energy \underline{released} while converting a \underline{neutral isolated gaseous} atom into a negatively charged gaseous ion by addition of electron

			Unit: $kJ mol^-1$

			$\vec{{Period}}$ - increases; \it{Cl is highest, this property has a number of exceptions in both, periods and groups}

			Group $\downarrow$ - decreases

		\item Electronegativity: tendency of an atom in a \underline{molecule} to attract the shared pair of electrons towards itself

			Unit: none (dimensionless property)
			$\vec{{Period}}$ - increases; \it{F is highest with 4.0, Cs is lowest with 0.7}

			Group $\downarrow$ - decreases
	\end{itemize}

	\subsection{Molecule}
	made of atoms, has independent existence, takes part in all reactions

	\subsubsection{Atomicity}
	number of atoms in a molecule

	\subsubsection{Caused by}
	Covalent bond (sharing of electrons) between two atoms

	\subsubsection{Properties}
	\begin{itemize}
		\item Electrical Conductor: only polar covalent molecules as electrolytes

		\item Magnetic: depends on atoms

		\item Solubility: depends on atoms, refer solubility section
	\end{itemize}

	\subsection{The Periodic Table}

	The periodic table has 18 Groups and 7 Periods. Elements of period 3 are called typical elements as they represent the general properties of their groups

	$G_1$ Alkali Metals: $H^1, Li^3, Na^{11}, K^{19}$

	$G_2$ Alkaline Earth Metals: $Be^4, Mg^{12}, Ca^{20}$

	$G_3$ - $G_{12}$: Transition elements, properties change periodically

	$G_{13}$ Boron Family: $B^5 (Metalloid), Al^{13}$

	$G_{14}$ Carbon Family: $C^6, Si^{14} (Metalloid)$

	$G_{15}$ Nitrogen Family: $N^7, P^{15}$

	$G_{16}$ Oxygen Family or chalcogen: $O^8, S^{16}$

	$G_{17}$ Halogen: $F^9, Cl^{17}$

	$G_{18}$ Halogen: $He^2, Ne^{10}, Ae^{18}, Kr^{36}$


	\subsubsection{Finding the group and period of an element}
	Period: Number of shells

	Group: if (number of valence electrons) < 2 then (number of valence electrons + 10) else (number of valence electrons)

	\subsubsection{Periodicity}
	properties that reappear at regular intervals or have a gradual variation

	\subsubsection{Comparison of Alkali and Halogens}

	Alkali:
	\begin{itemize}
		\item shining white solid metal
		\item posses one valence electron
		\item forms positive ion
		\item good conductors of electricity
		\item react vigorously with water and acid
		\item reducing agents
	\end{itemize}

	Halogens:
	\begin{itemize}
		\item colored diatomic non metals
		\item posses seven valence electrons
		\item form negative ion
		\item non conductors of electricity
		\item \underline{generally} do not react with dilute acids and water
		\item oxidizing agents
	\end{itemize}

	\subsection{Chemical Bonding}
	A chemical bond may be defined as the force of attraction between any two atoms in a molecule to maintain \underline{stability}

	\subsubsection{Stability}
	electron arrangement of closest inert gas as they are stable and nonreactive

	\subsubsection{Electrovalent Bond | Ionic Bond}
	The chemical compounds formed as a result of the transfer of electrons from one atom of an element to another atom of another element

	\begin{itemize}
		\item Ion: charged particle formed due to gain or loss of one or more $e^-$

	Atom lose $e^-$ to form electropositive element (cation)

	Atom gain $e^-$ to form electronegative element (anion)

		\item Electrovalency: The number of electrons, an atom of an element loses or gains to form an electrovalent bond is called its electrovalency

		\item Conditions:
			\begin{itemize}
				\item Low Ionization Potential: easily lose $e^-$, cations form easily
				\item High electron affinity: anion forms easily
				\item Large electronegativity difference
			\end{itemize}

		\item The metals of $G_{1,2,13}$ combine with $G_{15,16,17}$. CsF is the most ionic compound

		\item Bonds between metals and non-metals and ionic \underline{except} hydrogen
	\end{itemize}

	\subsubsection{Electron Dot symbol | Lewis Symbol}
	Generally, the valence $e^-$ of first element is marked by a '.' while that of second element by 'x'.

	\subsubsection{Covalent Bond | Molecular Bond}
	The chemical bond that is formed by mutual sharing of one or more pairs of electrons

	\begin{itemize}
		\item Types: Single, Double, Triple
		\item Covalency: number of electrons in formation of shared pair
		\item Non-polar compounds: Shared pair is equally distributed, formed in similar atoms (H-H, Cl-Cl, O=O) or when electronegativity difference between dissimilar atoms is little ($CH_4, CCl_4$). Do not ionize in waster due to lack of charge separation
		\item Polar Compound | Dipole molecule: Shared pair is not equally distributed, fractional charges develop (HCl, $H_2O$). Ionise in water or solution. It has both slight positive and slight negative charge, hence known as a dipole molecule
		\begin{itemize}
			\item Ionization: The fractional charges in solution are converted to complete charges and ions are produced
		\end{itemize}

		\item Conditions:
			\begin{itemize}
			\item four or more $e^-$ in valence shell \underline{except} H, He, B, Al, etc
			\item High electronegativity
			\item High electron affinity
			\item High Ionization Potential
		\end{itemize}
	\item Hydrogen can combine with $G_{14-17}$ with covalent bond
	\end{itemize}

	\subsubsection{Coordinate Bond | Dative Bond | Co-Ionic Bond}
	The bond formed by sharing of a pair of electrons, provided entirely by one of the atoms but shared by both

	Conditions:
		\begin{itemize}
				\item One compound has lone pair: a pair of $e^-$ not shared by any other atom
				\item One compound is short of pair
		\end{itemize}

	\subsubsection{Elements with more than one type of bond}
	\begin{itemize}
		\item Ammonium Chloride ($NH_4Cl$) has all three bonds
		\item Electrovalent and Covalent: $NaOH$, $CaCO_3$
		\item Covalent and Coordinate: $H_2SO_4$
	\end{itemize}

	\subsubsection{Formation of Hydroxl ($OH^-$) Ion}
	\begin{equation}
		H_2O \to H^+ + OH^-
	\end{equation}

	\subsubsection{Self Ionization of water}
	Eq:
		$H_2O \to H^+ + OH^-$

		$H^+ + H_2O \rightleftarrows H_3O^+$

		$H_2O + H_2O \rightleftarrows H_3O^+ + OH^-$

	\section{Acids, Bases and Salts}

\end{document}
